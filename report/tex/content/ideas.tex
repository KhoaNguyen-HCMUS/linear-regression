\section{Ý tưởng thực hiện}
Mục tiêu của đề tài là xây dựng mô hình dự đoán Student Performance (Performance Index) từ các đặc trưng hành vi – học tập, đồng thời so sánh nhiều phương án mô hình hóa để chọn mô hình gọn nhẹ, dễ diễn giải và có sai số thấp.

\subsection{Yêu cầu 1: Thực hiện phân tích khám phá dữ liệu}
Để thực hiện phân tích khám phá dữ liệu, chúng ta sẽ sử dụng các hàm thống kê mô tả và trực quan hóa dữ liệu để hiểu rõ hơn về các đặc trưng của dữ liệu. Các bước thực hiện được tham khảo từ tài liệu \cite{EDA_purpose} và \cite{EDA_step_by_step} bao gồm:
\begin{itemize}
	\item Tải dữ liệu từ file CSV.
	\item Thống kê nhanh số dòng, số cột, đặc trưng và kiểu dữ liệu của từng đặc trưng.
	\item Thống kê số lượng giá trị bị thiếu (null), duy nhất (unique) trong từng đặc trưng.
	\item Thống kê số lượng bộ dữ liệu bị trùng
	\item Làm sạch dữ liệu: loại bỏ hoặc thay thế các giá trị thiếu, xử lý ngoại lệ, loại bỏ các bản ghi trùng nhau
	\item Phân tích đơn biến nhằm nắm bắt phân phối của từng biến, phát hiện ngoại lệ và xác định biến cần biến đổi hoặc chuẩn hóa.
	\item Phân tích hai biến nhằm tìm hiểu mối quan hệ giữa các biến với biến mục tiêu (Performance Index).
	\item Phân tích tương tác và mối quan hệ phức tạp giữa các biến nhằm phát hiện các mẫu và mối liên hệ không tuyến tính.
	\item Trực quan hóa mối quan hệ giữa các đặc trưng và biến mục tiêu (Performance Index) bằng biểu đồ scatter plot, box plot, heatmap.
\end{itemize}

\subsection{Yêu cầu 2a: Sử dụng toàn bộ 5 đặc trưng để xây dựng mô hình}
Để xây dựng mô hình dự đoán Student Performance (Performance Index), chúng ta sẽ sử dụng toàn bộ 5 đặc trưng đã được phân tích ở trên. Các bước thực hiện bao gồm:
\begin{itemize}
	\item Huấn luyện 1 lần duy nhất cho 5 đặc trưng trên cho toàn bộ tập huấn luyện.
	\item Thể hiện công thức cho mô hình hồi quy tuyến tính với 5 đặc trưng.
	\item Đánh giá mô hình: sử dụng chỉ số MSE để đánh giá hiệu suất của mô hình trên tập kiểm tra.
\end{itemize}

\subsection{Yêu cầu 2b: Xây dựng mô hình sử dụng duy nhất 1 đặc trưng, tìm mô hình cho kết quả tốt nhất}

Để tìm mô hình tốt nhất sử dụng duy nhất 1 đặc trưng, chúng ta sẽ thực hiện các bước sau:
\begin{itemize}
	\item Thử nghiệm trên toàn bộ (5) đặc trưng đề bài cung cấp.
	\item Xáo trộn dữ liệu
	\item Sử dụng k-fold cross-validation (k=5) để đánh giá mô hình.
	\item Báo cáo kết quả MSE từ k-fold cross-validation của từng mô hình.
	\item Chọn mô hình có kết quả tốt nhất dựa trên chỉ số MSE.
	\item Tìm công thức hồi quy tuyến tính cho mô hình tốt nhất.
	\item Đánh giá mô hình: sử dụng chỉ số MSE để đánh giá hiệu suất của mô hình trên tập kiểm tra.
\end{itemize}

\subsection{Yêu cầu 2c: Sinh viên tự xây dựng/thiết kế mô hình, tìm mô hình cho kết quả tốt nhất}

Để tự xây dựng/thiết kế mô hình, ta sẽ dựa vào các đặc trưng của dữ liệu đã được khám phá ở Yêu cầu 1. Ý tưởng là sử dụng các đặc trưng có mối quan hệ mạnh với biến mục tiêu (Performance Index) để xây dựng mô hình. Các mô hình được xây dựng là:

\subsubsection{Mô hình 1:}
Dựa trên phân tích Count Plot và heatmap, tìm ra các đặc trưng có ảnh hưởng lớn nhất đến biến mục tiêu (Performance Index). Sử dụng các đặc trưng này để xây dựng mô hình hồi quy tuyến tính đơn giản:


\subsubsection{Mô hình 2:}
Từ phân tích Count Plot và heatmap, loại bỏ đặc trưng có ảnh hưởng nhỏ nhất và thêm vào bình phương của đặc trưng ảnh hưởng lớn nhất để tạo ra mô hình có độ chính xác cao hơn:
\subsubsection{Mô hình 3:}
Sử dụng đặc trưng tương tác giữa các đặc trưng để tạo ra mô hình có độ chính xác cao hơn. Cụ thể, ta sẽ sử dụng tích của các đặc trưng có mối quan hệ mạnh với biến mục tiêu (Performance Index) để tạo ra mô hình hồi quy tuyến tính.


